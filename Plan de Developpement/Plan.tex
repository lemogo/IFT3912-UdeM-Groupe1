\documentclass[10pt,a4paper]{report}
\usepackage[utf8]{inputenc}
\usepackage[francais]{babel}
\usepackage[T1]{fontenc}
\usepackage{amsmath}
\usepackage{amsfonts}
\usepackage{amssymb}
\usepackage{qtree}
\usepackage[left=1cm,right=1cm,top=1cm,bottom=1cm]{geometry}
		\author{Nicolas Lavoie Drapeau \and Alexandre Mathon-Roy \and Luc-Antoine Girardin \and William Tchoudi \and Lionnel Lemogo}
		\title{IFT3912 - Développement et maintenance,\\Plan de développement\\Équipe 1}
		\begin{document}
		\maketitle
		\begin{flushleft}
				\section*{Résumé}
					Permettre à des utilisateurs de créer ou de s'inscrire en ligne à 
					des événements reliés à l'entraînement et au maintient d'un rythme de vie sain.\\
				\section*{Work Breakdown Structure}
					\begin{enumerate}
						\item[1.] Analyse des besoins.\\
						\begin{enumerate}
							\item[1.1] Analyser les fonctions requises\\
							\item[1.2] Définir les besoins des interfaces web\\
							\item[1.3] Développer l'architecture du système web\\
							\item[1.4] D�crire les cas d'utilisation\\
						\end{enumerate}
				\item[2.] Analyse des risques.\\
					\begin{enumerate}
						\item[2.1] Identifier et analyser les risques.\\
						\item[2.2] Composer le plan d'aversion des risques.\\
					\end{enumerate}
				\item[3.] Qualité et métriques.\\
					\begin{enumerate}
						\item[3.1] Définir les métriques du projet.\\
						\item[3.2] Définir l'évaluation de qualité du logiciel.\\
					\end{enumerate}
				\item[4.] Conception.\\
					\begin{enumerate}
						\item[4.1] Conception des interfaces web.\\
						\item[4.2] Conception du pont entre web et le système d'origine (Conception d�tailler du pont Java-Web, UML )\\
						\item[4.3] Conception de l'architecture du projet.\\
						\item[4.4] Conception de la fonctionnalité de recherche.\\
						\item[4.5] Conception de la base donnée.\\
					\end{enumerate}
				\item[5.] Implémentation et tests unitaires.\\
					\begin{enumerate}
						\item[5.1] Coder la base de données.\\
						\item[5.2] Coder les interfaces statiques Web et faire des tests unitaires.\\
						\item[5.3] Coder la gestion dynamique des pages Web et faire les tests unitaires.\\
						\item[5.4] Coder le pont et faire des tests unitaires.\\
				\end{enumerate}
				\item[6.] Installation et déploiement.\\
					\begin{enumerate}
						\item[6.1] Installer le serveur web.\\
						\item[6.2] Installer la base de données.\\
						\item[6.3] Assurer la communication entre le pont, la base de données et le serveur.\\
						\item[6.4] Tester le serveur.\\
						\item[6.5] Tester le fonctionnement du pont.\\
					\end{enumerate}
				\end{enumerate}
				\section*{Échéancier et assignation des ressources}
						Taille de l'équipe: 5 membres.\\
						Expert Web: 2\\
						Expert DB: 1\\
						Expert java: 2\\
						Modèle de l'équipe: % contr�le décentralisé
				\\
				\section*{Analyse des risques}
						\begin{enumerate}
						\item[$\bullet$] Risque relié projet
						\begin{enumerate}
						\item[1.] Risque de dépasser la date limite. (Chance: 80\%, Impact: $\frac{3}{4}$)\\
						\item[2.] Problèmes de disponibilité du personnel. (Chance 60\%, Impact: $\frac{2}{4}$)\\
						\item[3.] Problèmes de compétence du personnel. (Chance 50\%, Impact: $\frac{1}{4}$)\\
						\item[4.] Risque de bris matériel. (Chance: 20\%, Impact: $\frac{2}{4}$)\\
				\end{enumerate}
				\item[$\bullet$] Risque techniques
					\begin{enumerate}
					
						\item[1.] Risque de documentation de mauvaise qualité. (Chance: 85\%, Impact: $\frac{1}{4}$)\\
						\item[2.] Risque sur qualit� du logiciel d�velopp�  (Chance: 65\%, Impact: $\frac{2}{4}$)\\
						\item[3.] Risque sur la taille du projet  (Chance: 75\%, Impact: $\frac{2}{4}$)\\
						\item[3.] Risque sur la  technonologie utilis�e (Chance: 65\%, Impact: $\frac{2}{4}$)\\
					\end{enumerate}
					\end{enumerate}
		\end{flushleft}
\end{document}